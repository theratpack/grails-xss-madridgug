% ====================================================================
%
% XSS Countermeasures in Grails.
%
% 1/10/2013 at Madrid GUG
%
% by The Rat Pack (Rafael Luque and Jose San Leandro)
%
%%%%%%%%%%%%%%%%%%%%%%%%%%%%%%%%%%%%%%%%%%%%%%%%%%%%%%%%%%%%%%%%%%%%%%

\documentclass{beamer}

\usetheme[coverbackgroundimage=images/6144568756_764c08d72c_o]{osoco2012}

\usepackage[english]{babel}
% or whatever. E.g. For English papers:
% \usepackage[english]{babel}

\usepackage[latin1]{inputenc}
\usepackage{times}
\usepackage[T1]{fontenc}
\usepackage{multimedia}
\usepackage{hyperref}
\usepackage{listings}

%% Presentation specific packages
\usepackage{pbsi}  % BrushScriptX-Italic font
\usepackage{fancybox}
\usepackage{soul}
\usepackage{color}
\usepackage{minted}
\usepackage{tikz}
\usetikzlibrary{arrows,shapes}

%% Paper Metadata
%%%%%%%%%%%%%%%%%%%%%%%%%%%%%%%%%%%%%%%%%%%%%%%%%%%%%%%%%%%%%%%%%%%%%%

\title{XSS Countermeasures in Grails}
%% \subtitle{An introduction for Java developers}
\author{
  Rafael Luque \href{http://twitter.com/rafael_luque}{@rafael\_luque} --- \href{http://osoco.es}{\bfseries OSOCO}\\
  Jos� San Leandro \href{http://twitter.com/rydnr}{@rydnr} --- \href{http://www.ventura24.es}{\bfseries Ventura24}\
}
\institute{%
  \pgfuseimage{madridgug-logo}
}
\date[10/2013]{October 2013}
\subject{Grails XSS prevention}
\keywords{Grails, XSS, security, exploit, botnet}

% Graphics
%%%%%%%%%%%%%%%%%%%%%%%%%%%%%%%%%%%%%%%%%%%%%%%%%%%%%%%%%%%%%%%%%%%%%%

% An image declaration
% E.g. by-nc-sa Creative Commons icon
% \pgfdeclareimage[width=2cm]{by-nc-sa}{images/by-nc-sa}

\pgfdeclareimage[width=.6cm]{grails-logo}{images/grails-logo}
\pgfdeclareimage[width=1.5cm]{madridgug-logo}{images/madridgug}
\pgfdeclareimage[width=\paperwidth, height=\paperheight]{hieroglyphs}{images/3996794712_37fc0788aa_o.jpg}
\pgfdeclareimage[width=\paperwidth, height=\paperheight]{grails-org-defacement}{images/grails-org-defacement}
\pgfdeclareimage[width=.9\textwidth]{grails-org-defacement-medium}{images/grails-org-defacement}
\pgfdeclareimage[width=\paperwidth, height=\paperheight]{white-rabbit}{images/5849593440_4bc8d03325_o.jpg}
\pgfdeclareimage[width=.9\textwidth]{hooked-browsers}{images/hooked-browsers-beef}
\pgfdeclareimage[width=.9\textwidth]{preparing-exploit-server}{images/preparing-exploit-server}
\pgfdeclareimage[width=.9\textwidth]{injecting-invisible-frame}{images/injecting-invisible-frame}
\pgfdeclareimage[width=.9\textwidth]{open-meterpreter-session}{images/open-meterpreter-session}
\pgfdeclareimage[width=.5\textwidth]{notepad-process}{images/notepad-process}
\pgfdeclareimage[width=.9\textwidth]{remote-shell}{images/remote-shell}
\pgfdeclareimage[width=.9\textwidth]{keylogging}{images/keylogging}
\pgfdeclareimage[width=.9\textwidth]{vnc}{images/vnc}
\pgfdeclareimage[width=.9\textwidth]{vnc-tightvnc}{images/vnc-tightvnc}
\pgfdeclareimage[width=.9\textwidth]{install-malware}{images/install-malware}
\pgfdeclareimage[width=.9\textwidth]{botnet-panel}{images/botnet-panel}
\pgfdeclareimage[width=\paperwidth, height=\paperheight]{zombies}{images/4986897433_aa757bdc83_o.jpg}


% If you want the table of contents to pop up at the beginning of each
% section:
%\AtBeginSection[]
%{
%{
%  \begin{frame}<beamer>[plain]{Contents}
%    \tableofcontents[currentsection,hideallsubsections]
%  \end{frame}}
%}

% If you wish to uncover everything in a step-wise fashion, uncomment
% the following command:

%\beamerdefaultoverlayspecification{<+->}

\setbeamertemplate{background}{}  %% Don't use OSOCO frame background

% Custom commands and configurations
%%%%%%%%%%%%%%%%%%%%%%%%%%%%%%%%%%%%%%%%%%%%%%%%%%%%%%%%%%%%%%%%%%%%%%

\newcommand{\handwritten}[1]{{\bsifamily\color{red}#1}}

\renewcommand{\theFancyVerbLine}{\sffamily \textcolor[rgb]{0.5,0.5,1.0}{\tiny \oldstylenums{\arabic{FancyVerbLine}}}}
\definecolor{srcbg}{rgb}{0.98,0.98,0.98}
\usemintedstyle{trac}

% Document contents
%%%%%%%%%%%%%%%%%%%%%%%%%%%%%%%%%%%%%%%%%%%%%%%%%%%%%%%%%%%%%%%%%%%%%%

\begin{document}
\tikzstyle{every picture}+=[remember picture]

% Contents

%% Title Page
%%%%%%%%%%%%%%%%%%%%%%%%%%%%%%%%%%%%%%%%%%%%%%%%%%%%%%%%%%%%%%%%%%%%%%
\begin{frame}[plain]
  %%\titlepage

  \vskip-1pt

  \hbox{\hskip-1cm
    \begin{tikzpicture}
	  \useasboundingbox (0,0) rectangle (\the\paperwidth, \the\paperheight) ;
	  \pgftext[at=\pgfpoint{0cm}{0cm},left,base]{\pgfuseimage{coverBackgroundImage}} ;
    \end{tikzpicture}
  }

  \vskip-1\paperheight

  % Title
  \usebeamercolor[fg]{title in cover}
  \hbox{\hskip-.01\paperwidth
    \vbox{
      \vskip.1\paperheight
      \usebeamerfont{title}\Huge\inserttitle\par
    }
  }

  % The optional subtitle
  \ifx\insertsubtitle\vskip0pt plus 1filll\else%
  \hbox{\hskip-.01\paperwidth
    \vbox{
      \vskip.05\paperheight
      \usebeamerfont{subtitle}\insertsubtitle\par
    }
  }
  \fi

  % Grails logo
  \vskip.24\paperheight
  \hbox{\hskip2.5cm
    \begin{pgfpicture}
      \pgflowlevel{\pgftransformrotate{62}}
      \pgfuseimage{grails-logo}
    \end{pgfpicture}
  }

  \vskip.25\paperheight

  % Author
  \ifx\insertauthor\@empty\else%
  \hbox{\hskip-.01\paperwidth
    \vbox{
      \usebeamerfont{author}\insertauthor\par
    }
  }
  \fi

  % Institution
  \ifx\insertinstitute\@empty\else%
  \vskip-0.17\paperheight%
  \hbox{\hskip.75\paperwidth
    \vbox{
      \usebeamerfont{institute}\insertinstitute\par
    }
  }
  \fi

  \vskip0pt plus 1filll
\end{frame}

% Grails.org XSS Demo
%%%%%%%%%%%%%%%%%%%%%%%%%%%%%%%%%%%%%%%%%%%%%%%%%%%%%%%%%%%%%%%%%%%%%%

\section{XSS Demo}

\subsection{Grails.org defacement}

{
\setbeamertemplate{background canvas}{\pgfuseimage{grails-org-defacement}}
\begin{frame}[plain]
  \vspace{-6cm}
  \begin{center}
    \LARGE\bfseries
    http://goo.gl/UGdJ0I
  \end{center}
\end{frame}
}

% XSS Intro
%%%%%%%%%%%%%%%%%%%%%%%%%%%%%%%%%%%%%%%%%%%%%%%%%%%%%%%%%%%%%%%%%%%%%%

\section{XSS Intro}

\begin{frame}[plain]
    \begin{center}
      \Huge\bfseries
      XSS Intro
    \end{center}
\end{frame}

\begin{frame}{XSS concepts}
  \begin{itemize}
    \item What's a XSS
    \item XSS Types: Reflected, stored, DOM-based.
    \item Famous XSS attacks: Samy worm, etc.
  \end{itemize}
\end{frame}

\begin{frame}{XSS threats}
  \begin{itemize}
    \item Interface defacement
    \item Session hijacking
    \item Click hijacking
    \item Malware infection
    \item Your PC may be joined to the horde of zombies in a BotNet.
  \end{itemize}
\end{frame}

% XSS Demo continuation: Following the white rabbit
%%%%%%%%%%%%%%%%%%%%%%%%%%%%%%%%%%%%%%%%%%%%%%%%%%%%%%%%%%%%%%%%%%%%%%

\section{Following the white rabbit}

{
\setbeamertemplate{background canvas}{\pgfuseimage{white-rabbit}}
\begin{frame}[plain]
    \Huge\bfseries
    \hspace{6.5cm}
    \begin{minipage}{.3\textwidth}
      Following \\
      the \\
      white \\
      rabbit\ldots
    \end{minipage}
\end{frame}
}

\begin{frame}[plain]
  \begin{center}
    \LARGE\bfseries
    Something more than a joke\ldots
  \end{center}

  \vspace{.5cm}

  \begin{center}
    \pgfuseimage{grails-org-defacement-medium}
  \end{center}
\end{frame}

\subsection{Hooking browser}

\begin{frame}[plain]
    \begin{center}
      \Huge\bfseries
      Hooking your browser
    \end{center}
\end{frame}

\begin{frame}[plain]
  \begin{center}
    \LARGE\bfseries
    Hooked browsers with BeEF
  \end{center}

  \vspace{.5cm}

  \begin{center}
    \pgfuseimage{hooked-browsers}
  \end{center}
\end{frame}

\subsection{Exploiting the system}

\begin{frame}[plain]
    \begin{center}
      \Huge\bfseries
      Exploiting your system
    \end{center}
\end{frame}

\begin{frame}[plain]
  \begin{center}
    \LARGE\bfseries
    Exploiting the browser
  \end{center}

  \vspace{.5cm}

  \begin{onlyenv}<1>
    1. Preparing the exploit server\ldots
    \begin{center}
      \pgfuseimage{preparing-exploit-server}
    \end{center}
  \end{onlyenv}

  \begin{onlyenv}<2>
    2. Injecting an invisible frame pointing to the exploit server\ldots
    \begin{center}
      \pgfuseimage{injecting-invisible-frame}
    \end{center}
  \end{onlyenv}

  \begin{onlyenv}<3>
    3. Exploit works and executes the payload\ldots
    \begin{center}
      \pgfuseimage{open-meterpreter-session}
    \end{center}
  \end{onlyenv}

  \begin{onlyenv}<4>
    4. Spawning notepad.exe process to migrate to\ldots
    \begin{center}
      \pgfuseimage{notepad-process}
    \end{center}
  \end{onlyenv}

\end{frame}

\subsection{Post-exploit phase}

\begin{frame}[plain]
    \begin{center}
      \Huge\bfseries
      Fun with post-exploitation
    \end{center}
\end{frame}

\begin{frame}[plain]
  \begin{center}
    \LARGE\bfseries
    Post-exploitation phase
  \end{center}

  \vspace{.5cm}

  \begin{onlyenv}<1>
    Run a remote shell
    \begin{center}
      \pgfuseimage{remote-shell}
    \end{center}
  \end{onlyenv}

  \begin{onlyenv}<2>
    Keylogging
    \begin{center}
      \pgfuseimage{keylogging}
    \end{center}
  \end{onlyenv}

%  \begin{onlyenv}<3>
%    Spy webcams
%    \begin{center}
%      \pgfuseimage{webcam}
%    \end{center}
%  \end{onlyenv}

  \begin{onlyenv}<3>
    Run VNC session
    \begin{center}
      \pgfuseimage{vnc}
    \end{center}
  \end{onlyenv}

  \begin{onlyenv}<4>
    Run VNC session
    \begin{center}
      \pgfuseimage{vnc-tightvnc}
    \end{center}
  \end{onlyenv}

\end{frame}

%\subsection{You're a zombie in my BotNet}
\subsection{Wake up! The matrix has you.}

{
\setbeamertemplate{background canvas}{\pgfuseimage{zombies}}
\begin{frame}[plain]
    \Huge\bfseries
    \vspace{2cm}
    \color{white}
    \begin{minipage}{.7\textwidth}
      Welcome to the\\ horde of\\ zombies
    \end{minipage}
\end{frame}
}

\begin{frame}[plain]
  \begin{center}
    \LARGE\bfseries
    Joining to a botnet
  \end{center}

  \vspace{.5cm}

  \begin{onlyenv}<1>
    1. Install the malware\ldots
    \begin{center}
      \pgfuseimage{install-malware}
    \end{center}
  \end{onlyenv}

  \begin{onlyenv}<2>
    2. Welcome to my botnet C\&C\ldots
    \begin{center}
      \pgfuseimage{botnet-panel}
    \end{center}
  \end{onlyenv}

\end{frame}


% Responsibilities
%%%%%%%%%%%%%%%%%%%%%%%%%%%%%%%%%%%%%%%%%%%%%%%%%%%%%%%%%%%%%%%%%%%%%%

\section{Responsibilities}

\begin{frame}[plain]
    \begin{center}
      \Huge\bfseries
      Responsibilities: Why is this still an issue?
    \end{center}
\end{frame}

\begin{frame}{Commercial software}
  \begin{itemize}[<+-| alert@+>]
    \item XSS is not known for business stakeholders
    \item For most people, security means attacking your servers
    \item Developers don't pay enough attention
  \end{itemize}
\end{frame}

\begin{frame}{Do your homework}
  \begin{itemize}[<+-| alert@+>]
    \item Raise awareness
    \item Practice with security tools
    \item Promote defensive coding
    \item Improve monitoring
  \end{itemize}
\end{frame}

% Grails encoding
%%%%%%%%%%%%%%%%%%%%%%%%%%%%%%%%%%%%%%%%%%%%%%%%%%%%%%%%%%%%%%%%%%%%%%

\section{Understanding Grails encoding}

{
\setbeamertemplate{background canvas}{\pgfuseimage{hieroglyphs}}
\begin{frame}[plain]
    \Huge\bfseries
    \vspace{-3.5cm}
    \begin{flushright}\color{white}Understanding Grails Encoding\end{flushright}
\end{frame}
}

% Grails pre-2.3
%%%%%%%%%%%%%%%%%%%%%%%%%%%%%%%%%%%%%%%%%%%%%%%%%%%%%%%%%%%%%%%%%%%%%%

\subsection{Grails pre-2.3}

\begin{frame}[plain]
    \begin{center}
      \Huge\bfseries
      Grails Pre-2.3 Gotchas
    \end{center}
\end{frame}

% Gotcha-1: Built-in default codec is none!

\begin{frame}[plain,fragile]

    \vspace{.5cm}

    \begin{center}
      \Huge \#1: Built-in default codec \\
      \visible<2>{\handwritten{is none!}}
    \end{center}

    \vspace{1cm}

    \begin{onlyenv}<1>
    \begin{center}
      \begin{minipage}{.9\textwidth}
        \begin{minted}[fontsize=\large,bgcolor=srcbg]{java}
grails.views.default.codec
        \end{minted}
      \end{minipage}
    \end{center}
    \end{onlyenv}

    \begin{onlyenv}<2>
    \begin{center}
      \begin{minipage}{.9\textwidth}
        \begin{minted}[fontsize=\large,bgcolor=srcbg]{java}
grails.views.default.codec = ''none''
        \end{minted}
      \end{minipage}
    \end{center}
    \end{onlyenv}

\end{frame}

\begin{frame}[plain,fragile]

    \vspace{.5cm}

    \begin{center}
      \Huge \#1: Built-in default codec \\
      \handwritten{is none!}
    \end{center}

    \vspace{1cm}

    \Large
    \textbf{Problems} \\[1em]

    \begin{onlyenv}<1>
      You have to escape explicitly every untrusted data:
      \begin{center}
        \begin{minipage}{.9\textwidth}
          \begin{minted}[fontsize=\large,bgcolor=srcbg]{java}
encodeAsHTML()
encodeAsJavaScript()
encodeAsURL()
          \end{minted}
        \end{minipage}
      \end{center}
    \end{onlyenv}

    \begin{onlyenv}<2>
      High likelihood of XSS vulnerabilities in production.\\[1em]
      E.g. Grails.org website.
    \end{onlyenv}

    \begin{onlyenv}<3>
      Double-encoding prevention over \emph{Security by default}.
    \end{onlyenv}

    \vfill

\end{frame}

\begin{frame}[plain,fragile]

    \vspace{.5cm}

    \begin{center}
      \Huge \#1: Built-in default codec \\
      \handwritten{is none!}
    \end{center}

    \vspace{1cm}

    \Large
    \textbf{Solution} \\[1em]

    \begin{onlyenv}<1>
      Change default codec to HTML:
      \begin{center}
        \begin{minipage}{.9\textwidth}
          \begin{minted}[fontsize=\large,bgcolor=srcbg]{java}
grails.views.default.codec = ''html''
          \end{minted}
        \end{minipage}
      \end{center}
    \end{onlyenv}

    \vfill

\end{frame}

% Gotcha-2: Inconsistent behaviour

\begin{frame}[plain,fragile]

  \tikzstyle{na} = [baseline=-.5ex]
  \vspace{.5cm}

  \begin{center}
    \Huge \#2: Inconsistent behaviour
  \end{center}

  \vspace{1cm}

  \begin{columns}
    \column{.5\textwidth}
    \tikz[baseline]{%
      \node[fill=green!20,ellipse,anchor=base] (b1)
           {Apply codec};
    }
    \column{.5\textwidth}
    \tikz[baseline]{%
      \node[fill=red!20,ellipse,anchor=base] (b2)
           {Does not apply codec};
    }
  \end{columns}

  \vspace{1cm}

  \begin{itemize}[<+-| alert@+>]
    \item GSP EL: \verb|${...}| \tikz[na] \node[coordinate] (a1) {};
    \item Tag: \verb|<g:tag .../>| \tikz[na] \node[coordinate] (a2) {};
    \item GSP EL in tag attribute: \verb|<g:tag a="${...}"/>| \tikz[na] \node[coordinate] (a3) {};
    \item Tag as a method: \verb|${g.tag(...)}| \tikz[na] \node[coordinate] (a4) {};
    \item Scriptlets: \verb|<%= ... %>| \tikz[na] \node[coordinate] (a5) {};
  \end{itemize}

  \begin{tikzpicture}[overlay]
    \path[->,green,line width=1.5pt]<2-> (a1) edge [out=90, in=-90] (b1);
    \path[->,red,line width=1.5pt]<3-> (a2) edge [out=0, in=-150] (b2);
    \path[->,red,line width=1.5pt]<4-> (a3) edge [out=0, in=-30] (b2);
    \path[->,green,line width=1.5pt]<5-> (a4) edge [out=45, in=0] (b1);
    \path[->,red,line width=1.5pt]<6-> (a5) edge [out=-15, in=-90] (b2);
  \end{tikzpicture}

\end{frame}

% Gotcha-3: Tag output is not escaped

\begin{frame}[plain,fragile]

  \vspace{.5cm}

  \begin{center}
    \Huge \#3: Tag output is not escaped
  \end{center}

  \vspace{1cm}

    \Large
    \textbf{Problems} \\[1em]

    \begin{onlyenv}<1>
      Review the tags you use to make sure they encode their output
      or have options for this (e.g. \verb|encodeAs| attribute).
    \end{onlyenv}

    \begin{onlyenv}<2>
      Review the tags from plugins you use.
    \end{onlyenv}

    \begin{onlyenv}<3>
      Review the tags you invoke as methods in Controllers.
    \end{onlyenv}

    \begin{onlyenv}<4>
      Don't trust Grails core tags, they have inconsistent behaviour. E.g:
      \begin{center}
        \begin{minipage}{.9\textwidth}
          \begin{minted}[fontsize=\large,bgcolor=srcbg]{java}
<g:fieldValue /> // HTML-encoded
<g:message />    // NO HTML-encoded
          \end{minted}
        \end{minipage}
      \end{center}
    \end{onlyenv}

    \vfill

\end{frame}


\begin{frame}[plain,fragile]

  \vspace{.5cm}

  \begin{center}
    \Huge \#3: Tag output is not escaped
  \end{center}

  \vspace{1cm}

    \Large
    \textbf{Solutions} \\[1em]

    \begin{onlyenv}<1>
      If tag implementation doesn't encode, add it explicitly or invoke it as a method inside a GSP expression:
      \begin{center}
        \begin{minipage}{.9\textwidth}
          \begin{minted}[fontsize=\large,bgcolor=srcbg]{java}
<g:message ... encodeAs=''HTML''/>
${g.message(...)}
g.message(...).encodeAsHTML()
          \end{minted}
        \end{minipage}
      \end{center}
    \end{onlyenv}

    \vfill

\end{frame}


%% Gotcha-4: GRAILS-7170

\begin{frame}[plain,fragile]

  \vspace{.5cm}

  \begin{center}
    \Huge \#4: g:message doesn't escape arguments
  \end{center}

  \vspace{1cm}

    \Large
    \textbf{Problems} \\[1em]

    \begin{onlyenv}<1>
      With default codec set to HTML the following XSS attack vector works:
      \begin{center}
        \begin{minipage}{.9\textwidth}
          \begin{minted}[fontsize=\footnotesize,bgcolor=srcbg]{java}
<g:message code='welcome' args='[params.user]'/>

where:
   welcome = Hi {0}!
   params.user = <script>alert('pwnd')</script>
          \end{minted}
        \end{minipage}
      \end{center}
    \end{onlyenv}

    \vfill

\end{frame}

\begin{frame}[plain,fragile]

  \vspace{.5cm}

  \begin{center}
    \Huge \#4: g:message doesn't escape arguments
  \end{center}

  \vspace{1cm}

    \Large
    \textbf{Solutions} \\[1em]

    \begin{onlyenv}<1>
      Upgrade to a Grails version with the issue (GRAILS-7170) fixed: \\[1em]
      2.0.5, 2.1.5, 2.2.2, 2.3-M1
    \end{onlyenv}

    \begin{onlyenv}<2>
      Escape explicitly or invoke the tag inside a GSP expression:
      \begin{center}
        \begin{minipage}{.9\textwidth}
          \begin{minted}[fontsize=\footnotesize,bgcolor=srcbg]{java}
<g:message code='welcome' args='[params.user]'
    encodeAs='HTML'/>

${g.message(code:'welcome', args:[params.user])}
          \end{minted}
        \end{minipage}
      \end{center}
    \end{onlyenv}

    \vfill

\end{frame}


% Gotcha-5: One codec is not enough

\begin{frame}[plain,fragile]

    \vspace{.5cm}

    \begin{center}
      \Huge \#5: One codec is not enough
    \end{center}

    \vspace{1cm}

    \begin{onlyenv}<1>
      You MUST use the escape syntax for the context of the HTML document you're putting untrusted data into:
      \begin{itemize}
        \item HTML
        \item JavaScript
        \item URL
        \item CSS
      \end{itemize}
    \end{onlyenv}

    \begin{onlyenv}<2>
      \textbf{HTML entity encoding doesn't work} if you're using untrusted data inside a <script>, or an event handler attribute like onmouseover, or inside CSS, or in a URL.
    \end{onlyenv}

\end{frame}


\begin{frame}[plain,fragile]

    \vspace{.5cm}

    \begin{center}
      \Huge \#5: One codec is not enough
    \end{center}

    \vspace{1cm}

    \Large
    \textbf{Problems} \\[1em]

    \begin{onlyenv}<1>
      You can override the default codec for a page, but not to switch the codec for each context:
      \begin{center}
        \begin{minipage}{.9\textwidth}
          \begin{minted}[fontsize=\large,bgcolor=srcbg]{java}
<%@page defaultCodec='CODEC' %>
          \end{minted}
        \end{minipage}
      \end{center}
    \end{onlyenv}

    \begin{onlyenv}<2>
      How to manage GSPs with mixed encoding requirements?
    \end{onlyenv}

    \vfill

\end{frame}


\begin{frame}[plain,fragile]

    \vspace{.5cm}

    \begin{center}
      \Huge \#5: One codec is not enough
    \end{center}

    \vspace{1cm}

    \Large
    \textbf{Solutions} \\[1em]

    \begin{onlyenv}<1>
      Turn off default codec for that page and use \verb|encodeAsJavaScript()|
      and \verb|encodeAsHTML()| explicitly everywhere.
    \end{onlyenv}

    \begin{onlyenv}<2>
      Extract the JavaScript fragment to a GSP tag encoding as JavaScript.
    \end{onlyenv}

    \vfill

\end{frame}


% Grails 2.3 encoding
%%%%%%%%%%%%%%%%%%%%%%%%%%%%%%%%%%%%%%%%%%%%%%%%%%%%%%%%%%%%%%%%%%%%%%

\subsection{Grails 2.3}

\begin{frame}[plain]
    \begin{center}
      \Huge\bfseries
      Grails 2.3 Encoding Enhancements
    \end{center}
\end{frame}

% Enhancement-1: New configuration more security by default

\begin{frame}[plain,fragile]

    \begin{center}
      \Huge \#1: New configuration more \emph{secure by default}
    \end{center}

\end{frame}

\begin{frame}[plain,fragile]

    \begin{center}
      \Huge \#1: New configuration more security by default
    \end{center}

    \begin{center}
      \begin{minipage}{1.1\textwidth}
        \begin{minted}[fontsize=\scriptsize,bgcolor=srcbg]{java}
grails {
  views {
    gsp {
      encoding = 'UTF-8'
      htmlcodec = 'xml' // use xml escaping instead of HTML4
      codecs {
        expression = 'html' // escapes values inside ${}
        scriptlet = 'html' // escapes output from scriptlets in GSPs
        taglib = 'none' // escapes output from taglibs
        staticparts = 'none' // escapes output from static templates
      }
    }
    // escapes all not-encoded output at final stage of outputting
    filteringCodecForContentType {
      //'text/html' = 'html'
    }
  }
}
        \end{minted}
      \end{minipage}
    \end{center}

\end{frame}

% Enhancement-2: Finer-grained control of codecs

\begin{frame}[plain,fragile]

    \begin{center}
      \Huge \#2: Finer-grained control of codecs
    \end{center}

    \vspace{1cm}

    \begin{onlyenv}<1>
      Control the codecs used per plugin:
      \begin{center}
        \begin{minipage}{\textwidth}
          \begin{minted}[fontsize=\scriptsize,bgcolor=srcbg]{java}
pluginName.grails.views.gsp.codecs.expression = 'CODEC'
          \end{minted}
        \end{minipage}
      \end{center}
    \end{onlyenv}

    \begin{onlyenv}<2>
      Control the codecs used per page:
      \begin{center}
        \begin{minipage}{\textwidth}
          \begin{minted}[fontsize=\normalsize,bgcolor=srcbg]{java}
<%@ expressionCodec='CODEC' %>
          \end{minted}
        \end{minipage}
      \end{center}
    \end{onlyenv}

    \begin{onlyenv}<3>
      Control the default codec used by a tag library:
      \begin{center}
        \begin{minipage}{\textwidth}
          \begin{minted}[fontsize=\normalsize,bgcolor=srcbg]{java}
static defaultEncodeAs = 'HTML'
          \end{minted}
        \end{minipage}
      \end{center}

      Or on a per tag basis:
      \begin{center}
        \begin{minipage}{\textwidth}
          \begin{minted}[fontsize=\normalsize,bgcolor=srcbg]{java}
static encodeAsForTags = [tagName: 'HTML']
          \end{minted}
        \end{minipage}
      \end{center}
    \end{onlyenv}

    \begin{onlyenv}<4>
      Add support for an optional \verb|encodeAs| attribute to all tags automatically:
      \begin{center}
        \begin{minipage}{\textwidth}
          \begin{minted}[fontsize=\normalsize,bgcolor=srcbg]{java}
<my:tag arg='foo.bar' encodeAs='JavaScript'/>
          \end{minted}
        \end{minipage}
      \end{center}
    \end{onlyenv}

\end{frame}


% Enhancement-3: Context-sensitive encoding switching

\begin{frame}[plain,fragile]

    \begin{center}
      \Huge \#3: Context-sensitive encoding switching
    \end{center}

    \vspace{1cm}

    \begin{onlyenv}<1>
      Tag \verb|withCodec('CODEC', Closure)| to switch the current default codec, pushing and popping a default codec stack.
      \begin{center}
        \begin{minipage}{\textwidth}
          \begin{minted}[fontsize=\normalsize,bgcolor=srcbg]{java}
out.println '<script type=''text/javascript''>'
withCodec(``JavaScript'') {
    out << body()
}
out.println()
out.println '</script>'
          \end{minted}
        \end{minipage}
      \end{center}
    \end{onlyenv}

    \begin{onlyenv}<2>
      Core tags like \verb|<g:javascript/>| and \verb|<r:script/>| automatically set an appropriate codec.
    \end{onlyenv}

\end{frame}


% Enhancement-4: Raw oputput

\begin{frame}[plain,fragile]

    \begin{center}
      \Huge \#4: Raw output
    \end{center}

    \vspace{1cm}

    \begin{onlyenv}<1>
      When you do not wish to encode a value, you can use the \verb|raw()| method.
      \begin{center}
        \begin{minipage}{\textwidth}
          \begin{minted}[fontsize=\normalsize,bgcolor=srcbg]{java}
${raw(book.title)}
          \end{minted}
        \end{minipage}
      \end{center}

      It's available in GSPs, controllers and tag libraries.
    \end{onlyenv}

\end{frame}


% Enhancement-5:

\begin{frame}[plain,fragile]

    \begin{center}
      \Huge \#5: Default encoding for all output
    \end{center}

    \vspace{1cm}

    \begin{onlyenv}<1>
      You can configure Grails to encode all output at the end of a response.
    \end{onlyenv}

    \begin{onlyenv}<2>
      \begin{center}
        \begin{minipage}{1.1\textwidth}
          \begin{minted}[fontsize=\scriptsize,bgcolor=srcbg]{java}
grails {
  views {
    gsp {
      codecs {
        expression = 'html' // escapes values inside ${}
        scriptlet = 'html' // escapes output from scriptlets in GSPs
        taglib = 'none' // escapes output from taglibs
        staticparts = 'raw' // escapes output from static templates
      }
    }
    // escapes all not-encoded output at final stage of outputting
    filteringCodecForContentType {
      'text/html' = 'html'
    }
  }
}
          \end{minted}
        \end{minipage}
      \end{center}

      If activated, the \verb|staticparts| codec needs to be set to raw so
      that static markup is not encoded.
    \end{onlyenv}

\end{frame}

% Grails Plugins Security
%%%%%%%%%%%%%%%%%%%%%%%%%%%%%%%%%%%%%%%%%%%%%%%%%%%%%%%%%%%%%%%%%%%%%%

\section{Grails Plugins Security}

\begin{frame}[plain]
    \begin{center}
      \Huge\bfseries
      Check your Plugins security
    \end{center}
\end{frame}

\begin{frame}[plain]{Plugins are also part of your application}
  \begin{itemize}[<+-| alert@+>]
    \item Grails plugins are not security audited
    \item Grails plugins are part of your application's attack surface
    \item Review plugins to make sure they encode, and if they don't you should JIRA the authors immediately, and fork and patch to fix your app quickly.
  \end{itemize}
\end{frame}

\begin{frame}[plain,fragile]{E.g. Javamelody vulnerability}
  \begin{itemize}[<+-| alert@+>]
    \item CVE-2013-4378 vulnerability reported.
    \item Allows \textbf{blind XSS} attack via \verb|X-Forwarded-For| header spoofing.
    \item The attack target is the admin's browser.
    \item Fixed in the last release (1.47).
    \item You should upgrade ASAP.
  \end{itemize}
\end{frame}

\begin{frame}[plain,fragile]{Demo: Javamelody XSSed}
  \begin{center}
    \pgfuseimage{javamelody-xssed}
  \end{center}
\end{frame}

% Solutions
%%%%%%%%%%%%%%%%%%%%%%%%%%%%%%%%%%%%%%%%%%%%%%%%%%%%%%%%%%%%%%%%%%%%%%

\section{Solutions}

\begin{frame}[plain]
    \begin{center}
      \Huge\bfseries
      Solutions: What options do we have?
    \end{center}
\end{frame}

\begin{frame}{Think like an attacker}
  \begin{itemize}[<+-| alert@+>]
    \item According to your grails version
    \item Find unescaped values
    \item Use fuzzers
    \item Read and understand Samy code
    \item Review OWASP XSS cheatsheets
  \end{itemize}
\end{frame}

\begin{frame}{Be aware}
  \begin{itemize}[<+-| alert@+>]
    \item Review your Grails app to double-check how all dynamic content gets escaped
    \item Monitor for suspicious traffic
    \item Spread the knowledge
    \item Adopt ZAP or similar fuzzers in your CI process
    \item Review available security plugins for Grails
  \end{itemize}
\end{frame}

\begin{frame}{Application firewalls}
 \begin{itemize}[<+-| alert@+>]
   \item Enable common, safe rules
   \item Log unexpected traffic
   \item Don't fool yourself
 \end{itemize}
\end{frame}

\begin{frame}{Early-adopt CSP}
 \begin{itemize}[<+-| alert@+>]
  \item CSP: Content Security Policy
  \item Adds headers to disable default behavior
   \begin{itemize}[<+-| alert@+>]
     \item inline Javascript
     \item dynamic code evaluation
   \end{itemize}
  \item Still a Candidate Recommendation of W3C
 \end{itemize}
\end{frame}

% Conclusions
%%%%%%%%%%%%%%%%%%%%%%%%%%%%%%%%%%%%%%%%%%%%%%%%%%%%%%%%%%%%%%%%%%%%%%

\section{Conclusions}

\begin{frame}[plain]
    \begin{center}
      \Huge\bfseries
      Conclusions: Grails can defeat XSS
    \end{center}
\end{frame}

\begin{frame}[plain]{Grails}
  \begin{itemize}[<+-| alert@+>]
    \item Is able to defend our application from XSS attacks
    \item But we need to pay attention to the details
    \item Upgrade to 2.3 ASAP
    \item Pay attention to XSS
  \end{itemize}
\end{frame}

\begin{frame}[plain]{XSS}
  \begin{itemize}[<+-| alert@+>]
    \item Is much more dangerous than defacement jokes
    \item The browsers are the actual target
    \item Difficult to monitor
    \item Unconfortable counter-measures in the browser: NoScript, Request Policy
  \end{itemize}
\end{frame}

\begin{frame}[plain]{Wake up}
 \begin{itemize}[<+-| alert@+>]
   \item Write secure applications by default
   \item Get yourself used with Metasploit, Burp, ZAP
   \item Spread the word both horizontally and vertically
 \end{itemize}
\end{frame}

\begin{frame}[plain]{Picture credits}
  \scriptsize
  \begin{itemize}
    \item Cover: \url{http://www.flickr.com/photos/usairforce/} CC by-nc
    \item White rabbit: \url{http://www.flickr.com/photos/alles-banane/5849593440} CC by-sa-nc
    \item Hieroglyphs: \url{http://www.flickr.com/photos/59372146@N00} CC by-sa-nc

  \end{itemize}
\end{frame}


% Fotos:
% http://www.flickr.com/photos/fling93/274632856/ CC by-sa-nc
% http://www.flickr.com/photos/rakka_pl/2332350845
% http://www.flickr.com/photos/uwehermann/132244825/
% http://www.flickr.com/photos/npobre/2601582256
% http://www.flickr.com/photos/dominiquegodbout/4286307359/
% http://www.flickr.com/photos/bartelomeus/4184705426/
% http://www.flickr.com/photos/grac/3729888371
% http://www.flickr.com/photos/dmosiondz/4293657601/
% http://www.flickr.com/photos/refractedmoments/65794219/
% http://www.flickr.com/photos/rtv/2269548635
% http://www.flickr.com/photos/49663222@N08/5682592501/


\end{document}
