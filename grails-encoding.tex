% Grails encoding
%%%%%%%%%%%%%%%%%%%%%%%%%%%%%%%%%%%%%%%%%%%%%%%%%%%%%%%%%%%%%%%%%%%%%%

\section{Understanding Grails encoding}

{
\setbeamertemplate{background canvas}{\pgfuseimage{hieroglyphs}}
\begin{frame}[plain]
    \Huge\bfseries
    \vspace{-3.5cm}
    \begin{flushright}\color{white}Understanding Grails Encoding\end{flushright}
\end{frame}
}

% Grails pre-2.3
%%%%%%%%%%%%%%%%%%%%%%%%%%%%%%%%%%%%%%%%%%%%%%%%%%%%%%%%%%%%%%%%%%%%%%

\subsection{Grails pre-2.3}

\begin{frame}[plain]
    \begin{center}
      \Huge\bfseries
      Grails Pre-2.3 Gotchas
    \end{center}
\end{frame}

% Gotcha-1: Built-in default codec is none!

\begin{frame}[plain,fragile]

    \vspace{.5cm}

    \begin{center}
      \Huge\color{red} \#1: Built-in default codec \\
      \visible<3>{\handwritten{is none!}}
    \end{center}

    \vspace{1cm}

    \begin{onlyenv}<2>
    \begin{center}
      \begin{minipage}{.9\textwidth}
        \begin{minted}[fontsize=\large,bgcolor=srcbg]{java}
grails.views.default.codec
        \end{minted}
      \end{minipage}
    \end{center}
    \end{onlyenv}

    \begin{onlyenv}<3>
    \begin{center}
      \begin{minipage}{.9\textwidth}
        \begin{minted}[fontsize=\large,bgcolor=srcbg]{java}
grails.views.default.codec = ''none''
        \end{minted}
      \end{minipage}
    \end{center}
    \end{onlyenv}

\end{frame}

\begin{frame}[plain,fragile]

    \vspace{.5cm}

    \begin{center}
      \Huge\color{red} \#1: Built-in default codec \\
      \handwritten{is none!}
    \end{center}

    \vspace{1cm}

    \Large
    \textbf{Problems} \\[1em]

    \begin{onlyenv}<1>
      You have to escape explicitly every untrusted data:
      \begin{center}
        \begin{minipage}{.9\textwidth}
          \begin{minted}[fontsize=\large,bgcolor=srcbg]{java}
encodeAsHTML()
encodeAsJavaScript()
encodeAsURL()
          \end{minted}
        \end{minipage}
      \end{center}
    \end{onlyenv}

    \begin{onlyenv}<2>
      High likelihood of XSS vulnerabilities in production.\\[1em]
      E.g. Grails.org website.
    \end{onlyenv}

    \begin{onlyenv}<3>
      Double-encoding prevention over \emph{Security by default}.
    \end{onlyenv}

    \vfill

\end{frame}

\begin{frame}[plain,fragile]

    \vspace{.5cm}

    \begin{center}
      \Huge\color{red} \#1: Built-in default codec \\
      \handwritten{is none!}
    \end{center}

    \vspace{1cm}

    \Large
    \textbf{Solution} \\[1em]

    \begin{onlyenv}<1>
      Change default codec to HTML:
      \begin{center}
        \begin{minipage}{.9\textwidth}
          \begin{minted}[fontsize=\large,bgcolor=srcbg]{java}
grails.views.default.codec = ''html''
          \end{minted}
        \end{minipage}
      \end{center}
    \end{onlyenv}

    \vfill

\end{frame}

% Gotcha-2: Inconsistent behaviour

\begin{frame}[plain,fragile]

  \tikzstyle{na} = [baseline=-.5ex]
  \vspace{.5cm}

  \begin{center}
    \Huge\color{red} \#2: Inconsistent behaviour
  \end{center}

  \vspace{1cm}

  \begin{columns}
    \column{.5\textwidth}
    \tikz[baseline]{%
      \node[fill=green!20,ellipse,anchor=base] (b1)
           {Apply codec};
    }
    \column{.5\textwidth}
    \tikz[baseline]{%
      \node[fill=red!20,ellipse,anchor=base] (b2)
           {Does not apply codec};
    }
  \end{columns}

  \vspace{1cm}

  \begin{itemize}[<+-| alert@+>]
    \item GSP EL: \verb|${...}| \tikz[na] \node[coordinate] (a1) {};
    \item Tag: \verb|<g:tag .../>| \tikz[na] \node[coordinate] (a2) {};
    \item GSP EL in tag attribute: \verb|<g:tag a="${...}"/>| \tikz[na] \node[coordinate] (a3) {};
    \item Tag as a method: \verb|${g.tag(...)}| \tikz[na] \node[coordinate] (a4) {};
    \item Scriptlets: \verb|<%= ... %>| \tikz[na] \node[coordinate] (a5) {};
  \end{itemize}

  \begin{tikzpicture}[overlay]
    \path[->,green,line width=1.5pt]<2-> (a1) edge [out=90, in=-90] (b1);
    \path[->,red,line width=1.5pt]<3-> (a2) edge [out=0, in=-150] (b2);
    \path[->,red,line width=1.5pt]<4-> (a3) edge [out=0, in=-30] (b2);
    \path[->,green,line width=1.5pt]<5-> (a4) edge [out=45, in=0] (b1);
    \path[->,red,line width=1.5pt]<6-> (a5) edge [out=-15, in=-90] (b2);
  \end{tikzpicture}

\end{frame}

% Gotcha-3: Tag output is not escaped

\begin{frame}[plain,fragile]

  \vspace{.5cm}

  \begin{center}
    \Huge\color{red} \#3: Tag output is not escaped
  \end{center}

  \vspace{1cm}

    \Large
    \textbf{Problems} \\[1em]

    \begin{onlyenv}<1>
      Review the tags you use to make sure they encode their output
      or have options for this (e.g. \verb|encodeAs| attribute).
    \end{onlyenv}

    \begin{onlyenv}<2>
      Review the tags from plugins you use.
    \end{onlyenv}

    \begin{onlyenv}<3>
      Review the tags you invoke as methods in Controllers.
    \end{onlyenv}

    \begin{onlyenv}<4>
      Don't trust Grails core tags, they have inconsistent behaviour. E.g:
      \begin{center}
        \begin{minipage}{.9\textwidth}
          \begin{minted}[fontsize=\large,bgcolor=srcbg]{java}
<g:fieldValue /> // HTML-encoded
<g:message />    // NO HTML-encoded
          \end{minted}
        \end{minipage}
      \end{center}
    \end{onlyenv}

    \vfill

\end{frame}


\begin{frame}[plain,fragile]

  \vspace{.5cm}

  \begin{center}
    \Huge\color{red} \#3: Tag output is not escaped
  \end{center}

  \vspace{1cm}

    \Large
    \textbf{Solutions} \\[1em]

    \begin{onlyenv}<1>
      If tag implementation doesn't encode, add it explicitly or invoke it as a method inside a GSP expression:
      \begin{center}
        \begin{minipage}{.9\textwidth}
          \begin{minted}[fontsize=\large,bgcolor=srcbg]{java}
<g:message ... encodeAs=''HTML''/>
${g.message(...)}
g.message(...).encodeAsHTML()
          \end{minted}
        \end{minipage}
      \end{center}
    \end{onlyenv}

    \vfill

\end{frame}


%% Gotcha-4: GRAILS-7170

\begin{frame}[plain,fragile]

  \vspace{.5cm}

  \begin{center}
    \Huge\color{red} \#4: g:message doesn't escape arguments
  \end{center}

  \vspace{1cm}

    \Large
    \textbf{Problems} \\[1em]

    \begin{onlyenv}<1>
      With default codec set to HTML the following XSS attack vector works:
      \begin{center}
        \begin{minipage}{.9\textwidth}
          \begin{minted}[fontsize=\footnotesize,bgcolor=srcbg]{java}
<g:message code='welcome' args='[params.user]'/>

where:
   welcome = Hi {0}!
   params.user = <script>alert('pwnd')</script>
          \end{minted}
        \end{minipage}
      \end{center}
    \end{onlyenv}

    \vfill

\end{frame}

\begin{frame}[plain,fragile]

  \vspace{.5cm}

  \begin{center}
    \Huge\color{red} \#4: g:message doesn't escape arguments
  \end{center}

  \vspace{1cm}

    \Large
    \textbf{Solutions} \\[1em]

    \begin{onlyenv}<1>
      Upgrade to a Grails version with the issue (GRAILS-7170) fixed: \\[1em]
      2.0.5, 2.1.5, 2.2.2, 2.3-M1
    \end{onlyenv}

    \begin{onlyenv}<2>
      Escape explicitly or invoke the tag inside a GSP expression:
      \begin{center}
        \begin{minipage}{.9\textwidth}
          \begin{minted}[fontsize=\footnotesize,bgcolor=srcbg]{java}
<g:message code='welcome' args='[params.user]'
    encodeAs='HTML'/>

${g.message(code:'welcome', args:[params.user])}
          \end{minted}
        \end{minipage}
      \end{center}
    \end{onlyenv}

    \vfill

\end{frame}


% Gotcha-5: One codec is not enough

\begin{frame}[plain,fragile]

    \vspace{.5cm}

    \begin{center}
      \Huge\color{red} \#5: One codec is not enough
    \end{center}

    \vspace{1cm}

    \begin{onlyenv}<1>
      You MUST use the escape syntax for the context of the HTML document you're putting untrusted data into:
      \begin{itemize}
        \item HTML
        \item JavaScript
        \item URL
        \item CSS
      \end{itemize}
    \end{onlyenv}

    \begin{onlyenv}<2>
      \textbf{HTML entity encoding doesn't work} if you're using untrusted data inside a <script>, or an event handler attribute like onmouseover, or inside CSS, or in a URL.
    \end{onlyenv}

\end{frame}


\begin{frame}[plain,fragile]

    \vspace{.5cm}

    \begin{center}
      \Huge\color{red} \#5: One codec is not enough
    \end{center}

    \vspace{1cm}

    \Large
    \textbf{Problems} \\[1em]

    \begin{onlyenv}<1>
      You can override the default codec for a page, but not to switch the codec for each context:
      \begin{center}
        \begin{minipage}{.9\textwidth}
          \begin{minted}[fontsize=\large,bgcolor=srcbg]{java}
<%@page defaultCodec='CODEC' %>
          \end{minted}
        \end{minipage}
      \end{center}
    \end{onlyenv}

    \begin{onlyenv}<2>
      How to manage GSPs with mixed encoding requirements?
    \end{onlyenv}

    \vfill

\end{frame}


\begin{frame}[plain,fragile]

    \vspace{.5cm}

    \begin{center}
      \Huge\color{red} \#5: One codec is not enough
    \end{center}

    \vspace{1cm}

    \Large
    \textbf{Solutions} \\[1em]

    \begin{onlyenv}<1>
      Turn off default codec for that page and use \verb|encodeAsJavaScript()|
      and \verb|encodeAsHTML()| explicitly everywhere.
    \end{onlyenv}

    \begin{onlyenv}<2>
      Extract the JavaScript fragment to a GSP tag encoding as JavaScript.
    \end{onlyenv}

    \vfill

\end{frame}


% Grails 2.3 encoding
%%%%%%%%%%%%%%%%%%%%%%%%%%%%%%%%%%%%%%%%%%%%%%%%%%%%%%%%%%%%%%%%%%%%%%

\subsection{Grails 2.3}

\begin{frame}[plain]
    \begin{center}
      \Huge\bfseries
      Grails 2.3 Encoding Enhancements
    \end{center}
\end{frame}

% Enhancement-1: New configuration more security by default

\begin{frame}[plain,fragile]

    \begin{center}
      \Huge\color{green} \#1: New configuration more \emph{secure by default}
    \end{center}

\end{frame}

\begin{frame}[plain,fragile]

    \begin{center}
      \Huge\color{green} \#1: New configuration more security by default
    \end{center}

    \begin{center}
      \begin{minipage}{1.1\textwidth}
        \begin{minted}[fontsize=\scriptsize,bgcolor=srcbg]{java}
grails {
  views {
    gsp {
      encoding = 'UTF-8'
      htmlcodec = 'xml' // use xml escaping instead of HTML4
      codecs {
        expression = 'html' // escapes values inside ${}
        scriptlet = 'html' // escapes output from scriptlets in GSPs
        taglib = 'none' // escapes output from taglibs
        staticparts = 'none' // escapes output from static templates
      }
    }
    // escapes all not-encoded output at final stage of outputting
    filteringCodecForContentType {
      //'text/html' = 'html'
    }
  }
}
        \end{minted}
      \end{minipage}
    \end{center}

\end{frame}

% Enhancement-2: Finer-grained control of codecs

\begin{frame}[plain,fragile]

    \begin{center}
      \Huge\color{green} \#2: Finer-grained control of codecs
    \end{center}

    \vspace{1cm}

    \begin{onlyenv}<1>
      Control the codecs used per plugin:
      \begin{center}
        \begin{minipage}{\textwidth}
          \begin{minted}[fontsize=\scriptsize,bgcolor=srcbg]{java}
pluginName.grails.views.gsp.codecs.expression = 'CODEC'
          \end{minted}
        \end{minipage}
      \end{center}
    \end{onlyenv}

    \begin{onlyenv}<2>
      Control the codecs used per page:
      \begin{center}
        \begin{minipage}{\textwidth}
          \begin{minted}[fontsize=\normalsize,bgcolor=srcbg]{java}
<%@ expressionCodec='CODEC' %>
          \end{minted}
        \end{minipage}
      \end{center}
    \end{onlyenv}

    \begin{onlyenv}<3>
      Control the default codec used by a tag library:
      \begin{center}
        \begin{minipage}{\textwidth}
          \begin{minted}[fontsize=\normalsize,bgcolor=srcbg]{java}
static defaultEncodeAs = 'HTML'
          \end{minted}
        \end{minipage}
      \end{center}

      Or on a per tag basis:
      \begin{center}
        \begin{minipage}{\textwidth}
          \begin{minted}[fontsize=\normalsize,bgcolor=srcbg]{java}
static encodeAsForTags = [tagName: 'HTML']
          \end{minted}
        \end{minipage}
      \end{center}
    \end{onlyenv}

    \begin{onlyenv}<4>
      Add support for an optional \verb|encodeAs| attribute to all tags automatically:
      \begin{center}
        \begin{minipage}{\textwidth}
          \begin{minted}[fontsize=\normalsize,bgcolor=srcbg]{java}
<my:tag arg='foo.bar' encodeAs='JavaScript'/>
          \end{minted}
        \end{minipage}
      \end{center}
    \end{onlyenv}

\end{frame}


% Enhancement-3: Context-sensitive encoding switching

\begin{frame}[plain,fragile]

    \begin{center}
      \Huge\color{green} \#3: Context-sensitive encoding switching
    \end{center}

    \vspace{1cm}

    \begin{onlyenv}<1>
      Tag \verb|withCodec('CODEC', Closure)| to switch the current default codec, pushing and popping a default codec stack.
      \begin{center}
        \begin{minipage}{\textwidth}
          \begin{minted}[fontsize=\normalsize,bgcolor=srcbg]{java}
out.println '<script type=''text/javascript''>'
withCodec(``JavaScript'') {
    out << body()
}
out.println()
out.println '</script>'
          \end{minted}
        \end{minipage}
      \end{center}
    \end{onlyenv}

    \begin{onlyenv}<2>
      Core tags like \verb|<g:javascript/>| and \verb|<r:script/>| automatically set an appropriate codec.
    \end{onlyenv}

\end{frame}


% Enhancement-4: Raw oputput

\begin{frame}[plain,fragile]

    \begin{center}
      \Huge\color{green} \#4: Raw output
    \end{center}

    \vspace{1cm}

    \begin{onlyenv}<1>
      When you do not wish to encode a value, you can use the \verb|raw()| method.
      \begin{center}
        \begin{minipage}{\textwidth}
          \begin{minted}[fontsize=\normalsize,bgcolor=srcbg]{java}
${raw(book.title)}
          \end{minted}
        \end{minipage}
      \end{center}

      It's available in GSPs, controllers and tag libraries.
    \end{onlyenv}

\end{frame}


% Enhancement-5:

\begin{frame}[plain,fragile]

    \begin{center}
      \Huge\color{green} \#5: Default encoding for all output
    \end{center}

    \vspace{1cm}

    \begin{onlyenv}<1>
      You can configure Grails to encode all output at the end of a response.
    \end{onlyenv}

    \begin{onlyenv}<2>
      \begin{center}
        \begin{minipage}{1.1\textwidth}
          \begin{minted}[fontsize=\scriptsize,bgcolor=srcbg]{java}
grails {
  views {
    gsp {
      codecs {
        expression = 'html' // escapes values inside ${}
        scriptlet = 'html' // escapes output from scriptlets in GSPs
        taglib = 'none' // escapes output from taglibs
        staticparts = 'raw' // escapes output from static templates
      }
    }
    // escapes all not-encoded output at final stage of outputting
    filteringCodecForContentType {
      'text/html' = 'html'
    }
  }
}
          \end{minted}
        \end{minipage}
      \end{center}

      If activated, the \verb|staticparts| codec needs to be set to raw so
      that static markup is not encoded.
    \end{onlyenv}

\end{frame}
